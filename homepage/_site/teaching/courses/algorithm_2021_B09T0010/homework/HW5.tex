% !Mode:: "TeX:UTF-8"
\documentclass[UTF8,16pt]{article} %文档声明  article,report,book

%导言区
\usepackage[space]{ctex} %引入宏包  
\usepackage{graphicx}
\usepackage{listings}
\usepackage{color}
\usepackage{lineno,hyperref,amsmath}
\usepackage[a4paper,left=35mm,right=35mm,top=45mm,bottom=15mm]{geometry} 

%opening
\title{《\heiti{算法设计与分析}》\heiti{第{\color{red}5}次作业}}
\author{\kaishu{姓名:}\underline{XXX} \quad\quad\quad\quad\quad  \kaishu{学号:}\underline{XXXXXXXX}}
\date{}

\begin{document}
	
\maketitle
\vbox{} %空行

\section*{\textbf{算法分析题}}
\noindent{\heiti{题目1}}:请论述回溯法和分支限界法的相同点与不同点。\\
{\heiti{答:}}\\
\rule[0pt]{14.3cm}{0.05em}

\vbox{} %空行
\noindent{\heiti{题目2}}:设某一机器由$n$个部件组成,每一种部件都可以从$m$个不同的供应商处购得,设$W_{ij}$ 是从供应商$j$ 处购得的部件$i$ 的重量,$C_{ij}$ 是相应的价格,试设计一个回溯法,给出总价格不超过$d$ 的最小重量机器设计。请描述算法的基本思想,要求画出解空间树,并给出相应的剪枝条件。试通过下面这个例子进行说明。
\begin{figure}[htbp]
	\centering
	\includegraphics[scale=0.45]{example.png}
\end{figure}\\
{\heiti{答:}}\\
\rule[0pt]{14.3cm}{0.05em}

\vbox{} %空行
\noindent{\heiti{题目3}}:认真阅读课本第五章5.9节和第六章6.7节旅行售货员问题,分别实现解旅行售货员问题的回溯算法和分支限界法。请给出两种算法的具体思想,实现两种算法,并填写下表,你需要填写你的算法在下列时间内能通过的实例数(n,即n个城市,自己制定距离),并附上运行结果截图。\\
% Please add the following required packages to your document preamble:
% \usepackage{graphicx}
% Please add the following required packages to your document preamble:
% \usepackage{graphicx}
\begin{table}[]
	\resizebox{\textwidth}{!}{%
		\begin{tabular}{|l|l|l|l|l|l|l|}
			\hline
			算法   & 10s & 20s & 30s & 50s & 1min & 2min \\ \hline
			回溯   &     &     &     &     &      &      \\ \hline
			分支限界 &     &     &     &     &      &      \\ \hline
		\end{tabular}%
	}
	\caption{通过实例数}
	\label{tab:my-table}
\end{table}
{\heiti{答:}}\\
\rule[0pt]{14.3cm}{0.05em}


\end{document}
