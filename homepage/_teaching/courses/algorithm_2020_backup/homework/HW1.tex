% !Mode:: "TeX:UTF-8"
\documentclass[UTF8,16pt]{article} %文档声明  article,report,book

%导言区
\usepackage[space]{ctex} %引入宏包
\usepackage{graphicx}
\usepackage{listings}
\usepackage{color}
\usepackage{lineno,hyperref,amsmath}
\usepackage[a4paper,left=35mm,right=35mm,top=45mm,bottom=15mm]{geometry}

%opening
\title{《\heiti{算法设计与分析}》\heiti{第{\color{red}1}次作业}}
\author{\kaishu{姓名:}\underline{XXX} \quad\quad\quad\quad\quad  \kaishu{学号:}\underline{XXXXXXXX}}
\date{}

\begin{document}
	
\maketitle
\vbox{} %空行

\section*{\textbf{算法分析题}}
\noindent{\heiti{题目1}}:假设$f(n)$和$g(n)$都是渐进非负函数。利用$\Theta$记号的基本定义来证明$max(f(n),g(n)) = \Theta(f(n) + g(n))$。\\
{\heiti{答:}}\\
\rule[0pt]{14.3cm}{0.05em}
\newline
%答案填写在这里

\noindent
\vbox{} %空行
\noindent{\heiti{题目2}}:求下列函数的渐进表达式:\\
(1) $3n^2+10n$ \quad\quad\  (2) $\frac{n^2}{10}+2^n$ \quad\quad\ (3) $10log3^n$  \quad\quad\ (4) $logn^3$\\
{\heiti{答:}}
\newline
%答案填写在这里

\noindent
\rule[0pt]{14.3cm}{0.05em}

\vbox{} %空行
\noindent{\heiti{题目3}}:对于下列各组函数$f(n)$和$g(n)$, 确定$f(n)=O(g(n))$或$f(n)=\Omega(g(n))$或$f(n)=\Theta(g(n))$, 并简述理由。\\
(1) $f(n)=logn^2;\quad\quad\quad\quad g(n)=logn+5$\\
(2) $f(n)=nlogn+n;\quad\quad\ g(n)=logn$\\
(3) $f(n)=logn^2;\quad\quad\quad\quad\ g(n)=\sqrt{n}$\\
(4) $f(n)=2^n;\quad\quad\quad\quad\quad\ g(n)=100n^2$\\
{\heiti{答:}}
\newline
%答案填写在这里

\noindent
\rule[0pt]{14.3cm}{0.05em}


%\vbox{} %空行
%\vbox{} %空行
%\section*{\textbf{算法实现题}}
%\noindent{\heiti{题目1}}:正整数 $x$ 的约数是能整除 $x$ 的正整数,其约数的个数记为$div(x)$,例如$div(10)=4$ 输入两个正整数 $a$ 和 $b$($a$小于等于 $b$),找出 $a$ 与 $b$ 之间约数个数最多的数 $x$。\\
%{\heiti{算法思路:}}\\
%{\heiti{结果截图:}}\\
%\rule[0pt]{14.3cm}{0.05em}
%
%\vbox{} %空行
%\noindent{\heiti{题目2}}:一本书的页码从自然数 1 开始顺序编码直到自然数 $n$。书的页码按照通常的习惯编排,每个页码都不含多余的前导数字 0。例如,第 6 页用数字 6 表示,而不是 06 或者 006 等。数字计数问题要求对给定书的总页码 $n$,计算出书的全部页码中分别用到多少次数字 0,1,2,3,…,9。\\
%{\heiti{算法思路:}}\\
%{\heiti{结果截图:}}\\

\end{document}
