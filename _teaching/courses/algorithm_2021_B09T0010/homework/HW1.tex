% !Mode:: "TeX:UTF-8"

\documentclass[UTF8,16pt]{article} %文档声明  article,report,book

%导言区
\usepackage[space]{ctex} %引入宏包
\usepackage{graphicx}
\usepackage{listings}
\usepackage{listings}
\usepackage{color}
\usepackage{lineno,hyperref,amsmath}
\usepackage[ruled,vlined]{algorithm2e}
\usepackage[a4paper,left=35mm,right=35mm,top=45mm,bottom=15mm]{geometry}
\lstset{language=C++}%这条命令可以让LaTeX排版时将C++键字突出显示

\lstset{breaklines}%这条命令可以让LaTeX自动将长的代码行换行排版

\lstset{extendedchars=false}%这一条命令可以解决代码跨页时,章节标题,页眉等汉字不显示的问题





%opening
\title{《\heiti{算法设计与分析}》\heiti{第{\color{red}1}次作业}}
\author{\kaishu{姓名:}\underline{XXX} \quad\quad\quad\quad\quad  \kaishu{学号:}\underline{XXXXXXXX}}
\date{}

\begin{document}
	
\maketitle
\vbox{} %空行

\section*{\textbf{算法分析题}}
\noindent{\heiti{题目1}}:以下算法的时间复杂度是:
\begin{algorithm}[htb]
    \SetAlgoNoLine
    \caption{Test(n)}
    \KwIn{上界 $n$}
    \KwOut{计算结果 $x$}
   $x\gets 2$\\
   \While{$x < \lfloor \frac{n}{2} \rfloor$}{$x=2*x$}
   \Return $x$
\end{algorithm}


\noindent
{\heiti{答:}}{答案写这里}\\
\rule[0pt]{14.3cm}{0.05em}


\vbox{} %空行
\noindent{\heiti{题目2}}:证明:$n!=o(n^{n}$)\\
{\heiti{证明:}}{答案写这里}\\
\rule[0pt]{14.3cm}{0.05em}

\vbox{} %空行
\noindent{\heiti{题目3}}:对于下列各组函数$f(n)$和$g(n)$, 确定$f(n)=O(g(n))$或$f(n)=\Omega(g(n))$或$f(n)=\Theta(g(n))$, 并简述理由。\\
(1) $f(n)=\log n^2;\quad\quad\quad g(n)=\sqrt{n}$\\
(2) $f(n)=n;\quad\quad\quad\quad\quad g(n)=\log^{2}n$\\
(3) $f(n)=10;\quad\quad\quad\quad\ g(n)=\log10$\\
(4) $f(n)=2^n;\quad\quad\quad\quad\ g(n)=3^n$\\
{\heiti{答:}}{答案写这里}\\
\rule[0pt]{14.3cm}{0.05em}


\vbox{} %空行
\noindent{\heiti{题目3}}:一本书的页码从自然数1开始顺序编码直到自然数$n$。书的页码按照通常的习惯编排,每个页码都不含多余的前导数字0。例如,第6页用数字6表示,而不是06或者006等。数字计数问题要求对给定书的总页码 $n$,计算出书的全部页码中分别用到多少次数字$0,1,2,3,\cdots,9$。请给出算法思路及伪代码,不需要写出完整代码{\bf(本题重点考查是否能清晰描述解题思路,以及是否会写伪代码)。}\\
{\heiti{算法思路:}}{答案写这里}\\
{\heiti{伪代码:}}\\
{答案写这里,使用latex的算法伪代码包}\\

\end{document}
