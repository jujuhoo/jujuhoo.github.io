\documentclass[UTF8,16pt]{article} %文档声明  article,report,book

%导言区
\usepackage{algorithm}  
\usepackage{algorithmicx}  
\usepackage{algpseudocode}  
\usepackage{amsmath}  
\usepackage{amssymb}
\renewcommand{\algorithmicrequire}{\textbf{Input:}}  % Use Input in the format of Algorithm  
\renewcommand{\algorithmicensure}{\textbf{Output:}} % Use Output in the format of Algorithm 
\usepackage[space]{ctex} %引入宏包  
\usepackage{graphicx}
\usepackage{listings}
\usepackage{color}
\usepackage{lineno,hyperref,amsmath}
\usepackage[a4paper,left=35mm,right=35mm,top=45mm,bottom=15mm]{geometry} 

%opening
\title{《\heiti{算法设计与分析}》\heiti{第{\color{red}2}次作业参考答案}}
%\author{\kaishu{姓名:}\underline{XXX} \quad\quad\quad\quad\quad  \kaishu{学号:}\underline{XXXXXXXX}}
\date{}

\begin{document}
	
\maketitle
\vbox{} %空行


\noindent{\heiti{题目1}}:求下列递推关系表示的算法复杂度。\\
(1)$T(n)=9T(\frac{n}{3}) + n$\\
(2)$T(n)=8T(n/6)+n3/2logn$\\
(3)$T(n)=7T(n/7)+n$\\
{\heiti{答:}}\\
(1)$T(n)=\Theta(n^2)$\\
由master定理:\\
$a=9,b=3,f(n)=n, n^{\log _ba}=n^2$\\
$\because f(n)=n=O(n^{(\log _ba)-\epsilon})$,即$\epsilon=1$\\ 
$\therefore T(n)=\Theta(n^{(\log _ba)})=\Theta(n^2)$\\
(2)$T(n)= \Theta (n^{\frac{3}{2}}\log n)$ \\
由master定理:\\
$a=8,b=6,f(n)=n^{\frac{3}{2}}\log n, n^{\log _ba}=n^{\log _68}$\\
$\because f(n)=O(n^{(\log _ba)+\epsilon})$\\
$\therefore T(n)=\Theta(n^{\frac{3}{2}}\log n)$ \\
(3)$T(n)=\Theta  (n\log n)$\\
由master定理:\\
$a=7,b=7,f(n)=n, n^{\log _ba}=n$\\
$\because f(n)=n=\Theta(n^{\log _ba})=\Theta(n)$\\
$\therefore T(n)=\Theta(n^{\log _ba}\log n)=\Theta(n\log n)$\\
\rule[0pt]{14.3cm}{0.05em}

\vbox{} %空行
\noindent{\heiti{题目2}}:设A[0:n-1]是一个元素个数为n的未排序的数组,运用分治算法找到第 k个最大的元素。请注意,你需要找的是数组排序后的第 k 个最大的元素,而不是第 k 个不同的元素。你需要给出具体的\textbf{算法思路}、\textbf{伪代码},并设计一个时间复杂度为$\mathbf{O(n)}$的算法(可以是平均或最坏情况。若能清晰写出最坏情况复杂度为$\mathbf{O(n)}$的算法,有额外分)。\\
{\heiti{答:}}\\
{\heiti{算法思路:}}

可以利用快速排序解决这个问题,先对原数组排序,再返回倒数第k个位置,这样平均时间复杂度是 $O(n \log n)$,但是还可以更快。

回顾一下快速排序,将数组$a[l:r]$划分成两个子数组$a[l:q - 1]$和$a[q + 1: r]$,每一轮划分都会确定一个元素的最终位置,假设元素$x$的最终位置是$q$,那么数组$a[l : q - 1]$ 中的每个元素小于等于 $a[q]=x$,且 $a[q]=x$ 小于等于 $a[q + 1 : r]$ 中的每个元素。

因此可以改进快速排序算法来解决这个问题:在分解的过程当中,对子数组进行划分,如果划分得到的 $q$ 正好就是需要的下标,就直接返回 $a[q]$;否则,如果 $q$ 比目标下标小,就递归右子区间,否则递归左子区间。这样就可以把原来递归两个区间变成只递归一个区间,提高效率。快速排序的性能和划分出的子数组的长度密切相关,最坏情况是$O(n^2)$,但是如果引入随机化的过程,期望的时间代价是$O(n)$。\\
{\heiti{伪代码:}}
\begin{algorithm}
	\caption{分治法求两有序数组的中位数}  
	\label{alg:Framwork}  
	
	\begin{algorithmic}[1]  
		\Require  
		数组 A, 整数 k;    
		\Ensure  
		数组 A中第 k大的元素;  
		\Function{partition}{$int~X[], int~l, int~r$}
		\State $x \gets X[r],~i \gets l-1$
		\For{$j = l \to r$}
		\If{$X[j] <= x$}
		\State $swap(X[++i],~a[j])$
		\EndIf
		\EndFor
		\State $swap(X[i+1],~a[j])$
		\State \Return $i+1$
		\EndFunction
		\Function {quickSelect}{$int~X[],int~l, int~r, int~index$}
		\State $q \gets l$;
		\If{$q == index$}
		\State \Return $X[q]$;
		\Else
		\State \Return $q < index?~ \Call{quickSelect}{X, q+1, r, index} : \Call{quickSelect}{X, l, q-1, index}$
		\EndIf
		\EndFunction
	\end{algorithmic}  
\end{algorithm}
\\
下面给出最坏时间复杂度为O(n)的算法$Select(A,k)$求第k小元素的思路(类似,同学们可思考求第k大元素的思路):\\
(1)将$n$个元素分为$\left \lceil \frac{n}{5}\right \rceil$组,每组5个元素,每组排序;{\color{red}代价$(O(n))$}\\
(2)将每组第3个元素取出,得到大小为$\left \lceil \frac{n}{5}\right \rceil$的数组$M$;\\
(3)最后一组中的元素可能不足5个(1个取出,2个取较小;3个取中间,4个取第二小);\\
(4)$m=Select(M, \left \lceil \frac{\left | M\right |}{2}\right \rceil)$;{\color{red}代价$T(\left \lceil \frac{n}{5}\right \rceil)$}\\
(5)$A$中至少有$3(\left \lceil \frac{1}{2}\left \lceil \frac{n}{5}\right \rceil\right \rceil-2) \geq \frac{3n}{10}-6$个元素大于等于$m$;\\
(6)类似地,$A$中至少有$\frac{3n}{10}-6$个元素小于$m$;\\
(7)依次扫描整个数组;($A[i]<m$时放入$A_{1}$, $A[i]=m$时放入$A_{2}$,$A[i]>m$时放入$A_{3}$);{\color{red}代价$(O(n))$}\\
(8)当$k\leq |A_1|$时,调用$Select(A_1,k)$;($A_1$大小最多为$n-(\frac{3n}{10}-6)=\frac{7n}{10}+6$,{\color{red}代价$(T(\frac{7n}{10}+6))$})\\
(9)当$|A_1|< k \leq |A_1|+|A_2|$时,返回$m$;\\
(10)当$k>|A_1|+|A_2|$时,调用$Select(A_3, k-|A_1|-|A_2|)$;($A_3$大小最多为$n-(\frac{3n}{10}-6) = \frac{7n}{10}+6$);{\color{red}代价$(T(\frac{7n}{10}+6))$}\\
复杂度:$$T(n)=T(\left \lceil \frac{n}{5}\right \rceil)+T(\frac{7n}{6})+O(n)\leq O(n)$$



\vbox{} %空行

\noindent{\heiti{题目3}}:动手设计并实现一个可以进行两个$n$位大整数的乘法运算的算法。你需要给出具体的\textbf{算法思路}、\textbf{伪代码},并对你设计的算法进行\textbf{复杂度分析},除此之外,你还需要给出\textbf{算法运行结果截图},并用你熟悉的图形库\textbf{画出输入规模$n$与运行时间的关系图}。\\
{\heiti{答:}}答案写在这里\\
\rule[0pt]{14.3cm}{0.05em}
{\heiti{算法思路:}}\\  \href{https://www.cnblogs.com/little-kwy/archive/2017/09/30/7613642.html}{参考此blog}\\
{\heiti{伪代码:}
	\begin{algorithm}
		\caption{分治法求大整数乘法(假设X和Y都是n位十进制整数)}  
		\label{alg:Framwork}  
		
		\begin{algorithmic}[1]  
			\Require  
			大整数 X, 大整数 Y, 长度 n;    
			\Ensure  
			X和Y相乘的乘积;  
			\Function{Sign}{$long~A$}
			\State \Return $A>0?~1:-1$
			\EndFunction
			\Function{calculateSame}{$long~X,~ long~Y,~int~n$}
			\State $sign \gets \Call{Sign}{X} * \Call{Sign}{Y}$
			\State $X=abs(X)$
			\State $Y=abs(Y)$
			\If{$X==0~||~Y==0$}
			\State \Return 0
			\ElsIf{$n==1$}
			\State \Return $sign * X * Y$
			\Else
			\State $A=(X/10^{\frac{n}{2}})$
			\State $B=(X \%10^{\frac{n}{2}})$
			\State $C=(Y/10^{\frac{n}{2}})$
			\State $D=(Y\%10^{\frac{n}{2}})$
			\State $AC=\Call{calculateSame}{A,C,~\frac{n}{2}}$
			\State $BD=\Call{calculateSame}{B,D,~\frac{n}{2}}$
			\State $ABCD=\Call{calculateSame}{(A-B),(D-C),~\frac{n}{2}}+AC+BD$
			\State \Return $sign*(AC*10^{n} + ABCD * 10^{\frac{n}{2}} + BD)$
			\EndIf
			\EndFunction
		\end{algorithmic}  
	\end{algorithm}

	\begin{algorithm}
	\caption{分治法求大整数乘法(假设X和Y位数不相等)}  
	\label{alg:Framwork}  
	
	\begin{algorithmic}[1]  
		\Require  
		大整数 X, 大整数 Y, X位数 xn, Y位数 yn;    
		\Ensure  
		X和Y相乘的乘积;  
		\Function{calculateUnSame}{$long~X,~ long~Y,~int~xn,~int~yn$}
		\If{$X==0~||~Y==0$}
		\State \Return 0
		\ElsIf{$((xn==1~\&\&~yn==1) ~||~xn==1~||~yn==1)$}
		\State \Return $X * Y$
		\Else
		\State $xn0=xn/2,~yn0=yn/2$
		\State $xn1=xn-xn0,~yn1=yn-yn0$
		\State $A=(X/10^{xn0})$
		\State $B=(X \%10^{xn0})$
		\State $C=(Y/10^{yn0})$
		\State $D=(Y\%10^{yn0})$
		\State $AC=\Call{calculateUnSame}{A,C,~xn1,~yn1}$
		\State $BD=\Call{calculateUnSame}{B,D,~xn0,~yn0}$
		\State $ABCD=\Call{calculateUnSame}{(A*10^{xn0}-B), (D-C*10^{yn0}), xn1, yn1}$
		\State \Return $2*AC*10^{xn0+yn0}+ABCD+2*BD$
		\EndIf
		\EndFunction
	\end{algorithmic}  
\end{algorithm}
	

\begin{figure}[bh]
	\includegraphics[width=1.0\textwidth]{3.png}
	\caption{(图为刘诺昂同学所画关系图(供参考))}
\end{figure}

}

\end{document}
